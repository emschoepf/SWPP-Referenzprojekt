\chapter*{Eidesstattliche Erklärung}
Ich erkläre an Eides statt, dass ich die vorliegende Diplomarbeit selbst verfasst und keine anderen als die angeführten Behelfe verwendet habe. Alle Stellen, die wörtlich oder inhaltlich den angegebenen Quellen entnommen wurden, sind als solche kenntlich gemacht.
Ich bin damit einverstanden, dass meine Arbeit öffentlich zugänglich gemacht wird.

\vspace{1cm}
\begin{tabular}{c c c}
	& \hspace{4cm} & \\\cline{1-1}
	Ort, Datum & & \\
	\vspace{2cm}
	& & \\\cline{1-1}\cline{3-3}
	Maximilian Hikel & & Emanuel Götsch\\ 

\end{tabular}

\chapter*{Abnahmeerklärung}
Hiermit bestätigt der Auftraggeber, dass das übergebene Produkt dieser Diplomarbeit den dokumentierten Vorgaben entspricht. Des Weiteren verzichtet der Auftraggeber auf unentgeltliche Wartung und Weiterentwicklung des Produktes durch die Projektmitglieder bzw. die Schule.

\vspace{1cm}
\begin{tabular}{c}
	\\\cline{1-1}
	Ort, Datum\\
	\vspace{2cm}
	\\\cline{1-1}
	Auftraggeber
\end{tabular}	

\chapter*{Vorwort}
Unser Refferenzprojekt "GeoQuest" ist ein Grundlegenes Programm/eine Grundlegende App, bei welcher man jeden Brunnen in ganz Imst nicht nur finden und Orten kann, sondern man kann auch Bilder zu jedem eigenen Brunnen anschauen.
Wir bedanken uns auch sehr bei der Hilfe unserer Klassenkameraden, be der Verwirklichung des Projeks, um dies auf GitHub bearbeiten zu können.


\chapter*{Abstract (Deutsch)}
(ca. ½ bis max. 2 Seiten)
Kurzbeschreibung von Aufgabenstellung und Problemlösung.

\chapter*{Abstract (Englisch)}
(ca. ½ bis max. 2 Seiten)